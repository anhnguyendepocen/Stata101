\documentclass{tufte-handout}

%\geometry{showframe} % display margins for debugging page layout

\usepackage[utf8]{inputenc}
\usepackage[french]{babel}
\usepackage{graphicx} % allow embedded images
  \setkeys{Gin}{width=\linewidth,totalheight=\textheight,keepaspectratio}
  \graphicspath{{graphics/}} % set of paths to search for images
\usepackage{amsmath}  % extended mathematics
\usepackage{manfnt}   % Knuth's like warning
\usepackage{booktabs} % book-quality tables
\usepackage{units}    % non-stacked fractions and better unit spacing
\usepackage{multicol} % multiple column layout facilities
\usepackage{lipsum}   % filler text
\usepackage{fancyvrb} % extended verbatim environments
  \fvset{fontsize=\normalsize}% default font size for fancy-verbatim environments
\usepackage{gitinfo}  % version tracking
\usepackage[french=guillemets*]{csquotes} 
\frenchspacing
\DecimalMathComma

\definecolor{pastil}{rgb}{0.67,0,0}
\definecolor{grey30}{rgb}{0.3,0.3,0.3}
\definecolor{grey50}{rgb}{0.5,0.5,0.5}
\definecolor{code}{rgb}{0,0.42,0.33}

\hypersetup{colorlinks=true,
            urlcolor=pastil,
            citecolor=,
            pagecolor=pastil,
            filecolor=,
            linkcolor=}

\newenvironment{explain}
{\begin{itshape}\small\par\noindent\marginpar{\dbend}\hskip-.1ex}%
{\end{itshape}}

% Standardize command font styles and environments
\newcommand{\doccmd}[2][]{%
  \texttt{#2}%
  \ifthenelse{\isempty{#1}}%
    {% add the command to the index
      \index{#2 command@\protect\texttt{#2}}% command name
    }%
    {% add the command and package to the index
      \index{#2 command@\protect\texttt{#2} (\texttt{#1} package)}
    }%
}
\newcommand{\docpkg}[1]{\texttt{#1}\index{packages!#1@\texttt{#1}}}% package name

\usepackage{makeidx}
\makeindex

\title{Découverte du logiciel Stata\thanks{Programme disponible sur le site \href{http://ritme.com/fr/training/stata}{www.ritme.com}.}}
\author[chl]{Christophe Lalanne}
\date{} % without \date command, current date is supplied


\begin{document}

\maketitle% this prints the handout title, author, and date

\begin{abstract}
\noindent
Les exercices proposés dans ce document ont pour but d'illustrer et
d'approfondir les notions présentées durant la formation. Les exemples présentés
ont été testés avec Stata 13 (MP) et 14 (SE) mais devraient fonctionner avec les
versions > 11. Il est conseillé d'enregistrer les commandes Stata dans un \enquote{script
  \texttt{do}}.
\end{abstract}

\section{Les commandes à retenir}\label{sec:memo}

La commande \doccmd{describe} (voir également \doccmd{codebook}) fournit des
informations essentielles sur une structure de données.

\section{Applications}\label{sec:application}

\begin{enumerate}
\item (a) Charger les données \texttt{smoke.dta} \cite{juul14} et étudier la
  structure de données (type de variables, nombre d'observations). (b) Quelle
  est la proportion d'hommes dans cet échantillon ? (c) Quelle est l'étendue
  pour la variable \texttt{age} ? (d) Quelle est l'âge moyen (avec écart-type)
  des hommes et des femmes ? (e) Vérifier que le nombre total de cigarettes
  fumées par jour n'est > 0 que pour les fumeurs actuels. (f) On définit
  l'indice de masse corporelle comme le rapport poids (en kg) sur taille (en m)
  au carré ; créer une nouvelle variable \texttt{bmi} représentant l'IMC des
  participants à cette étude.
\end{enumerate}


\bibliography{refs}
\bibliographystyle{plainnat}

% Colophon
\vspace*{\fill}
\begin{flushright}
\begin{footnotesize}
Stata MP version 13.1 (Copyright 1985-2013 StataCorp LP).\\
Ce document a été préparé à l'aide de \LaTeX.\\
Version \gitVtags : \gitAbbrevHash{} (\gitAuthorDate).
\end{footnotesize}
\end{flushright}

\printindex

\end{document}
